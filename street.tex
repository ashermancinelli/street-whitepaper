
%Preamble
    \documentclass[12pt]{article}
    \usepackage{setspace}
    \usepackage{arxiv}
    \usepackage{epigraph}
    \usepackage{cite}
    \setlength{\parindent}{4em}
    \setlength{\parskip}{1em}
    \linespread{1}

    
\newcommand{\epigraphIT}[1]{
    \epigraph{\textit{#1}}
}

\newcommand{\editonlyfn}[1]{
    \ifdefined\isedit
    \footnote{#1}
    \fi
}

\def \qFirstCorTwelveTwentyEight {\epigraphIT{And God has appointed in the church first apostles, second prophets, third teachers, then miracles, then gifts of healing, helping, administrating, and various kinds of tongues.}{1 Corinthians 12:28}}

\def \qJamesOneTwentySeven {\epigraphIT{Religion that is pure and undefiled before God the Father is this: to visit orphans and widows in their affliction, and to keep oneself unstained from the world.}{{James 1:27}}}

\def \qJamesTwoFive {\epigraphIT{Listen, my beloved brothers, has not God chosen those who are poor in the world to be rich in faith and heirs of the kingdom, which he has promised to those who love him?}{{James 2:5}}}

\def \qGreatCommission {\epigraphIT{Go therefore and make disciples of all nations, baptizing them in the name of the Father and of the Son and of the Holy Spirit, teaching them to observe all that I have commanded you.}{{Matthew 28:19}}}

\def \qPiperCoronary {\epigraphIT{[Adrenaline] doesn't do much for Mondays. I am more thankful for my heart. It just keeps on serving\dots Coronary Christians are like the heart in the causes they serve\dots O, for coronary Christians! Christians committed to great Causes, not great comforts.}{John Piper\cite{coronary}}}

\def \qLeastOfThese {\epigraphIT{Truly I tell you, whatever you did for one of the least of these brothers and sisters of mine, you did for me.}{{Matthew 25:40}}}

\def \qJamesPhysical {\epigraphIT{If one of you says to them, “Go in peace; keep warm and well fed,” but does nothing about their physical needs, what good is it?}{{James 2:16}}}

\def \qTwoOrMoreAreGathered {\epigraphIT{For where two or three are gathered in my name, there am I among them.}{{Matthew 18:20}}}

\def \qLeaders {\epigraphIT{Let the greatest among you become as the youngest, and the leader as one who serves.}{{Luke 22:25-26}}}

\def \qChristianHedonism {\epigraphIT{God is most glorified in us when we are most satisfied in him\dots our pursuit of joy in him is essential.}{John Piper\cite{piper_1995}}}

\def \qHebSix {\epigraphIT{For God is not unjust so as to overlook your work and the love that you have shown for His name in serving the saints, as you still do.}{{Hebrews 6:10}}}

\def \qFirstThesFive {\epigraphIT{Now may the God of peace himself sanctify you completely, and may your whole spirit and soul and body be kept blameless at the coming of our Lord Jesus Christ.}{{1 Thessalonians 5:23}}}

\def \qJohnEdwardsJoy {\epigraphIT{God made the world that He might communicate, and the creature receive, His glory; and that it might [be] received both by the mind and heart.}{Jonathan Edwards\cite{taylor_piper_2004}}}

\def \qMattJesusSumm {}

\def \qGalaiansForgetPoor {}


    \title{Street Ministry \\
        \large Equipping the Church \\
        \large DRAFT}

    \author{Asher Mancinelli \\
        Indian Trail Church \\
        Spokane, WA 99218 \\
        \texttt{asher.mancinelli@pnnl.gov} \\ }
%/Preamble

% WB: maybe better motivate service through scripture
% invite others into the ministry is good
% how specifically do I plan to move the gospel forward through this
% not remove people from poverty, though that is a goal, but to alleviate eternal suffering
% how do we do this as the church?

\begin{document}
\maketitle

\begin{abstract}
    % LF: local church not contributing to systemic problems, not hindering other groups already at work
    % LF: maybe give more information about other groups working in the area, give other resources for how to go about This
    % LF: If there is any part of this that is lacking, it is where the church is falling short
    \textit{
    By equipping the saints to lead and minister in the context of Spokane, Washington, we as the Church may better live out the Gospel.
    The local church's ministry to the homeless and food-insecure in Spokane is best accomplished by small teams with one leader, working in coordination with other long-term organizations (e.g. Union Gospel Mission\footnote{https://www.uniongospelmission.org/}, the House of Charity\footnote{https://www.cceasternwa.org/}, etc) already at work.
    The local church should not contribute to systemic problems or hinder the work of full-time ministries, but should contribute to the work of long-term organizations.
    The local church should use its financial resources to tend to the poor, and its knowledge of the Gospel to bring more people into a saving knowledge of Christ.
    }
\end{abstract}

\keywords{
    \textbf{The Church:} All those who have trusted in Christ for salvation, and who are incorporated into the Body of Christ by the Holy Spirit, spanning all time and space
    \and \textbf{local church:} The local church is a group of believers in a given area who are organized under leadership, and who regularly gather together to carry out the mission of the Church
    \and \textbf{The Gospel:} ``\textit{[T]hat Christ died for our sins in accordance with the Scriptures, that he was buried, that he was raised on the third day in accordance with the Scriptures}'' \footnote{1 Cor. 15:3-4}}

\clearpage

\section{Audience}

    This work targets churches in Spokane with a biblical understanding of the Gospel's call for the church to engage and come alongside the poor.
    I do not attempt here to provide an exhaustive theology of social justice, but rather the specific steps that local churches in Spokane can take to better care for the needy.
    This is intended for congregations that understand the need to engage with the poor, but perhaps lack the information or experience to carry it out.
    Note that all Bible passages are from the English Standard Version\cite{esv2016}.

\section{Opportunities to Equip the Saints}

    % I really want to convey here that there are issues in the church at the moment.
    % I want to be clear that there are some areas that I think we can and should grow in, what I think success looks like, and how I believe God is leading us there.
    % Problem -> Goal -> How we get there
    % In this pattern, I do not quite know how to organize the first section...
    % This section is especially open to reconstruction and correction
    In my experience with local churches, organizations and individuals in Spokane, believers tend to feel ill-equipped in several areas.
    Believers in middle- to upper-class Spokane tend to be somewhat removed from the needs of the poor.
    As a result, they do not know what physical needs they are able to meet, or what of those needs are of the highest priority.
    Believers also rarely feel comfortable sharing the Gospel, which is significant because of the opportunities to share the gospel downtown.
    The church can bridge this gap by equipping:
    \footnote{Editors: I intend to show (1) where the church is lacking, (2) what success looks like for the church, and (3) how we get from (1) to (2).
    In trying to convey this message, section (1) always feels problematic.
    I am not sure how best to articulate the current shortcomings of the church in the service to the poor, so this section is especially open to recommendation.}
    \begin{enumerate}
        \item Members with information needed to tend to the physical needs of the poor
          \footnote{``\textit{If a brother or sister is poorly clothed and lacking in daily food, and one of you says to them, “Go in peace, be warmed and filled,” without giving them the things needed for the body, what good is that?}'', James 2:15-16}
        \item Team members to share the Gospel and pray with people
          \footnote{``\textit{Go therefore and make disciples of all nations, baptizing them in the name of the Father and of the Son and of the Holy Spirit, teaching them to observe all that I have commanded you.}'', Matthew 28:19-20}
        \item Leaders to coordinate and empower team members
          \footnote{``\textit{[W]hoever would be great among you must be your servant, and whoever would be first among you must be your slave, even as the Son of Man came not to be served but to serve, and to give his life as a ransom for many}'', Matthew 20:26-28}
    \end{enumerate}
    This is the standard laid out by Scripture, and should the Church be anything less than this, She is falling short of the calling of the Church.
    \footnote{Editors: this sentence was added to clarify that the Church seems to be failing in these areas in some capacity---to give it some teeth, where before the list was all phrased as positives (e.g. ``this is where we can grow'')}


\qJamesOneTwentySeven

\section{Intended Results}

    All team members should be equipped in the following capacities:
    \begin{enumerate}
        \item Preparing and distributing items for physical aid
        \item Sharing the gospel
        \item Praying with/for the served
        \item Deescalating conversations with upset people
        \item Developing relationships with those being served
    \end{enumerate}

    Leaders should be equipped in the following additional capacities:
    \begin{enumerate}
        \item Navigating the streets and leading teams on foot
        \item Leading and comforting team members coming for the first time
        \item Equipping others to lead in the same capacities
    \end{enumerate}

\section{Implementation}

This section describes the steps that the local church can take to achieve the goals previously outlined.

\subsection{Team Members}

    % Define what it would look like for a nonbeliever or non church member to come and be ministered to
    The following applies to volunteers from within the church, however a visiting nonbeliever or non-member may not need to bear these responsibilities.
    The volunteer may serve along with the team and be ministered to by the rest of the team.
    In this way, God's character is revealed to both the team member and the ones being served; the ministry can be both inward and outward facing.
    Otherwise, every team member should be ready for the following five responsibilities:

\subsubsection{Catering to the Physical Needs}

    \qJamesPhysical
    Though the spiritual health is in every case more important than the physical, the physical is still of utmost importance, and we are commanded to tend to the physical health of those who lack it.
    The majority of Jesus' miracles were physical in nature, and Paul includes the body in the process of sanctification.
    \footnote{1 Thessalonians 5:23}
    About 15\% of Spokane County is within the poverty threshold as defined by the US Census Bureau\cite{census}.
    There are many ways that we as the local church can tend to the needs of our community.
    In my experience, the most common way for the local church to provide is through bagged lunches, hygiene supplies, and clothes.
    Historically, lunches have included a sandwich, a fruit, and a snack of some sort.
    It is worth noting that many homeless people do not have healthy teeth, so they cannot eat fruits like apples or crunchy snacks.
    Additionally, foods with nut products should be kept separate in case of a nut allergy.
    Of all the clothing items to donate, socks and underwear seem to be in the highest demand.
    This is because used underwear cannot be donated and many people are not comfortable giving them out.
    The needy do greatly appreciate when we overcome this social stigma and donate unused underwear, as fewer people are willing to do so.
    The organizations and businesses along the routes also appreciate handing out goods near trash cans and/or helping pick up garbage after handing out goods.

    \qHebSix
    Each member of the Body of Christ is gifted uniquely for the construction of the Kingdom of God, and each may contribute according to their gifting.
    For example, some may be physically incapable of walking long distances, so their contribution may be the purchasing or preparing of goods to be taken with the rest of the team.
    Each ought to contribute according to their capacity, however I have met many elderly ladies, exceeding eighty years of age and with many crippling physical ailments, who faithfully take to the streets with whatever aid they can provide.
    Their steadfastness has been an encouragement to me, and I hope it will also be for any members of the Body that feel incapable of contributing.
    Let it also be an encouragement that the Father will notice and remember the efforts of every member of the body---the valiant efforts of the physically impaired are of great value to the church, and are pleasing to the Father.

\subsubsection{Sharing the Gospel}

    \qGreatCommission
    The most capable entity in equipping believers to know and share the Gospel is the local church.
    The local church should equip its members to know and share the Gospel by biblical teaching from the pulpit, and through members teaching each other.
    Each member of a team going downtown ought be a member or attendee of a local church that professes Christ and teaches biblical doctrine, enabling the member to articulate the Gospel.
    This is a great opportunity for more mature members of the Body of Christ to model to the younger ones how to share the Gospel.
    Particularly, parents have the opportunity to show their children how to share the Gospel, a skill that may translate to their schools and eventually their workplaces.

\subsubsection{Prayer}

    In my experience, one of the methods of sharing the Gospel that has met the least friction is prayer.
    For example, ``Father in heaven, I thank you for your Gospel, which is\dots'' with a subsequent articulation of the Gospel.
    Prayer opens up a window for sharing the Gospel and is an opportunity to bring someone's needs to the Father, and every team member should be comfortable in praying with and for people they meet.

\subsubsection{Deescalation}

    Occasionally, people being served get agitated for one reason or another, and it is important to know how to deescalate these situations.
    For example, a member of a team I led once offered some cookies to a man with his dog.
    The man became quite upset that my team member only had cookies for people, and didn't bring dog treats.
    As he became more upset, my team member wisely replied, ``Well I am very sorry that we don't have any dog treats, but I will make sure to pack some next time!''
    She then proceeded to walk away from the man and wish him a nice day (she also did in fact make sure to bring dog treats the next time!).
    Any team member would be well-served to follow her pattern, which is to:
    \begin{enumerate}
        \item Speak softly, kindly and calmly,
            \footnote{Proverbs 25:21-22}
        \item empathize with what they seem to be feeling,
        \item apologize,
        \item and if the person continues to escalate, walk away
    \end{enumerate}
    Interactions tend to deescalate when the volunteer makes everyone feel heard, as many homeless people do not often feel that way.
    Additionally, this is an example of why it is always a good idea to walk with other members of the team.

\subsubsection{Developing Relationships}

    \qPiperCoronary
    It is in the context of relationship that the Holy Spirit seems to do the most work.
    \footnote{Matthew 18:20}
    In living out the Gospel, we as the Church seek not to simply distribute food to people, we seek to love them and forge relationships with them.
    This does not happen in a single day, but over weeks, months, and years.
    By going downtown many times, each team member ought to get to know and pray for specific people.
    This means remembering names and details about those we serve, with the intention of forming long-term relationships.
    Writing down names and praying over specific people, whether individually or in community, is a great way to remember details from interactions.
    This is builds relationships because remembering the names of people and details about them is indicative of intentionality and genuine care for the person.
    \footnote{Editors: The previous two sentences were in the ``Prayer'' section, but was moved here. Is this the best place for it? How can this thought also be incorporated in the prayer section?}
    Referring to a person by his or her name and remembering information about them carries a special significance, especially when this demographic is often treated as sub-human.

\subsection{Leaders}

    The responsibilities of team leaders exceed that of the other team members.
    Volunteers should be made aware that they may be subjected to outdated opinions, such as misogyny and racism.
    Occasionally, residents and people on the streets may speak or act out in such a way that volunteers, specifically women, are uncomfortable.
    \footnote{Editors: How do I make explicit the need for male leaders and the gender specific considerations of leaders?}
    Volunteer teams tend to feel more comfortable and interactions tend to stay more appropriate when there is at least one man on any team, and leaders should ensure this is the case.

\subsubsection{Leading on Foot}

    \qLeaders
    It is usually best to have a specific route in mind before actually hitting the streets for a couple of reasons.
    \footnote{Luke 14:28-33}
    First, if a couple of members of the team are talking for a particularly long time with someone, it may be helpful for the team to temporarily split up.
    In that case, there should be at least two people equipped to lead a group, so that no group is left without anyone to lead.
    Then, because there is already an established route, each group knows where the other is going, and can regroup later.
    Additionally, a team leader should always be conscious of where his or her team members are, ideally keeping all of them within eyesight.
    Some team members may absentmindedly walk away from the group and end up alone in a crowd of people, which is undesirable.
    There is usually not any danger, however it is the team leader's responsibility to make sure that everyone stays together and is accounted for.
    \par Leaders would also do well to watch for animals, as some can be aggressive or diseased, and some team members may be allergic to or afraid of animals.
    It is usually effective to stand in between the team members and the animal, so the team members feel comfortable.

\subsubsection{Leading First-Timers}

    When leading someone downtown who has never gone before, it is extremely helpful and reassuring to describe to them what typically happens and what the plan for that day is.
    Even if they seem comfortable, they may need some encouragement to feel ready to talk to people or hand out food.
    For example, as the leader you may begin an interaction with somebody and invite the newcomer into the conversation, or ask them to give some food to someone in particular to get them started.
    Encouragement, reassurance, and gratitude go a long way in making a team member feel comfortable sharing and speaking with people on the streets.

\subsubsection{Equipping Others to Lead}

    Leading is the best way to equip another person to lead.
    Especially if someone has participated several times before, it may be helpful to let that individual carry the food/goods and handle most of the interactions with people.
    This will boost their confidence and make them feel comfortable, while you stay close by and encourage their interactions.
    The mark of a truly experienced leader is their ability to empower others to lead---in the same way, a leader downtown ought to bring up other capable leaders from within their team.

\section{Concluding Notes}

    \qChristianHedonism
    I have attempted to show (1) where the church in Spokane is presently lacking in service to the poor, (2) what it looks like for the church to grow in those areas, and (3) the steps we can take to get from (1) to (2).
    This growth is not emotionless, however.
    Christ is faithful to provide us with surpassing joy and satisfaction in Him when we obey His word, and service to the poor is no exception.
    I encourage each local church in Spokane to engage with the poor of our community no matter how uncomfortable it may be.
    \par \textbf{Uncover the treasures of joy in store for those faithful to live out the Gospel.}
    \qJohnEdwardsJoy

\bibliography{references}{}
\bibliographystyle{unsrt}

\end{document}
