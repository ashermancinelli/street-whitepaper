\subsubsection{Leading on Foot}

    \qLeaders
    It is usually best to have a specific route in mind before actually hitting the streets for a couple of reasons.
    \footnote{Luke 14:28-33}
    First, if a couple of members of the team are talking for a particularly long time with someone, it may be helpful for the team to temporarily split up.
    In that case, there should be at least two people equipped to lead a group, so that no group is left without anyone to lead.
    Then, because there is already an established route, each group knows where the other is going, and can regroup later.
    Additionally, a team leader should always be conscious of where his or her team members are, ideally keeping all of them within eyesight.
    Some team members may absentmindedly walk away from the group and end up alone in a crowd of people, which is undesirable.
    There is usually not any danger, however it is the team leader's responsibility to make sure that everyone stays together and is accounted for.
    \par Leaders would also do well to watch for animals, as some can be aggressive or diseased, and some team members may be allergic to or afraid of animals.
    It is usually effective to stand in between the team members and the animal, so the team members feel comfortable.

