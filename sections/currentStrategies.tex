
\section{Current Strategies at Work}

    Although narrative is almost always more powerful than information in moving people to action, statistics and information are essential in developing strategy, especially when the group being served carries many stigmas. 

    Charity groups ought to equip themselves with the information necessary to best minister to those experiencing homelessness, and churches are no exception.
    Various organizations in Spokane go about this differently, and here we will examine the strategies of the two largest charity organizations in Spokane, after equipping ourselves with some foundational knowledge.

    \subsection{Foundational Knowledge}
        According to the Human Rights Campaign, Spokane has a score of 30 out of 30 with its anti-discrimination laws, but only a 2 out of 4 with regards to single-occupancy gender-neutral housing\cite{hrc}.
        Additionally, Spokane receives a score of 16 out of 28 when it comes to employment municipality.
        This means that there are additional boundaries in the way of queer people gaining employment in the city.
        Across the United States, queer people are about twice as likely as straight cisgender people to be unemployed (9\% vs 5\%), twice as likely to be food insecure (27\% vs 15\%), and 7\% more likely to have an income below \$24,000\cite{hrc}.
        These are all contributing factors to homelessness, and since Washington State is $6^{th}$ in the United States in its queer population, our churches and charity organizations must be equipped to serve them alongside straight people experiencing homelessness.
        With this information in mind, we will examine the strategies of the two biggest charity organizations operating in the Spokane area, and how their strategies may affect queer people, positively or negatively.

    \subsection{Catholic Charities}
        The Catholic Charities (CCEWa) has a housing-first model irrespective of the belief of the individual\cite{cceasternwa}:
        \par\textit{
            We serve people based on their need, not our creed.
            Everything from meals to emergency sleeping in shelter beds to quiet chapel or prayer and reflection time to respite care for those exiting the hospital to case management is provided in the shelter, along with a variety of supported services for health, education, job training, etc.
            HOC runs with both professional staff and hundreds of volunteers, including folks from many religious congregations in town.
            HOC also partners with other downtown providers of services to those experiencing homelessness, many of which are other religiously affiliated nonprofits and ministries.
        }
        \par This means that anyone who comes to their door that has not been previously flagged by the organization are welcome to sleep and seek rehabilitation.
        They will not turn any queer person away on that basis, however they are so often overfilled that sexual orientation is the least of the boundaries preventing them from getting housing assistance.
        This has both benefit and drawback; because they accept such broad group, they are very often flooded and overwhelmed with benefactors, yet many receive housing help that would not otherwise, such as queer people that are experiencing homelessness.
        In my experience with CCEWa, this has led to the discouragement of some of their leadership, as it is much harder to hold their recipients accountable and to track their progress.
        It seems that the most effective method for rehabilitation is partnership with groups of people that all hold each other accountable.
        This strategy has seen astounding results in the microfinance industry.
        For example, in India, a group that used the microfinance model achieved a repayment rate of 99.87\% on its microloans\cite{soni_2014}.
        The Union Gospel Mission's (UGM) model looks much more like this.

    \subsection{Union Gospel Mission}
        The Union Gospel Mission has a different philosophy, which centers on the growth and empowerment of individuals through four programs\cite{ugm}:
        \begin{itemize}
            \item Food and shelter
            \item Job training
            \item Recovery
            \item Youth outreach
        \end{itemize}
        
        \par Additionally, Phil Altmeyer (the regional UGM director) is planning on distributing a document to all regional churches and charity organizations to unify our efforts.
        This document, titled the \textit{Covenant of Compassion}, outlines 7 points which they hope all organizations may agree on, and around which we may all form our strategies:
        \begin{itemize}
            \item{Empower}
            \item{Compassion}
            \item{Work}
            \item{Sobriety}
            \item{Relationship}
            \item{Good Neighbor}
        \end{itemize}

        \par Again, this strategy has several drawbacks and several key benefits.
        Those who are fortunate enough to receive assistance through their programs are typically more closely attended to, and are held much more accountable.
        However, due to the risks of abuse and the lack of effective internal policing, queer people are not permitted to receive group housing services.
        Historically, abuse within shelters has been a difficult problem, as the prejudiced beliefs and behaviors of some of the recipients cannot be effectively filtered for, and shelters are often understaffed, which means there is likely not an effective security team to care for queer people in the shelter as attentively as is appropriate.
        Unfortunately this means 
