
\section{Current Strategies at Work}

    Charity groups ought to equip themselves with the information necessary to best minister to those experiencing homelessness, and churches are no exception.
    Various organizations in Spokane go about this differently, and here we will examine the strategies of the two largest charity organizations in Spokane after equipping ourselves with some foundational knowledge before examining the strategies I have been using.

    \subsection{Foundational Knowledge}
        According to the Human Rights Campaign, Spokane has a score of 30 out of 30 with its anti-discrimination laws, but only a 2 out of 4 with regards to single-occupancy gender-neutral housing\cite{hrc}.
        Additionally, Spokane receives a score of 16 out of 28 when it comes to employment municipality.
        This means that there are additional boundaries in the way of queer people gaining employment in the city.
        Across the United States, queer people are about twice as likely as straight cisgender people to be unemployed (9\% vs 5\%), twice as likely to be food insecure (27\% vs 15\%), and 7\% more likely to have an income below \$24,000\cite{hrc}.
        These are all contributing factors to homelessness, and since Washington State is $6^{th}$ in the United States in its queer population, our churches and charity organizations must be equipped to serve them alongside straight people experiencing homelessness.
        With this information in mind, we will examine the strategies of the two biggest charity organizations operating in the Spokane area, and how their strategies may affect queer people, positively or negatively.

    \input{sections/currentStrategies/cceasternwa}
    \input{sections/currentStrategies/ugm}
    \subsection{Strategy Summary}
        In conclusion, many organizations in Spokane are working continually to serve the portion of the queer community the is experiencing homelessness, yet there is continually more work to be done as well.
        From my experience, both downtown over the last three years and in my relationships with the Altmeyer's, I believe that Spokane will continue to improve in its support for queer people as long as we work together and are roughly on the same page with our strategies.
        Both the UGM and the Catholic Charities provide a diverse set of support structures to queer people experiencing homelessness, and these services continue to grow.
        However, it is apparent that progress still needs to be made, especially in the support of transgender people.
