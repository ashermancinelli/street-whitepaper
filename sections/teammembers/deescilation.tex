\subsubsection{Deescalation}

    Occasionally, people being served get agitated for one reason or another, and it is important to know how to deescalate these situations.
    For example, a member of a team I led once offered some cookies to a man with his dog.
    The man became quite upset that my team member only had cookies for people, and didn't bring dog treats.
    As he became more upset, my team member wisely replied, ``Well I am very sorry that we don't have any dog treats, but I will make sure to pack some next time!''
    She then proceeded to walk away from the man and wish him a nice day (she also did in fact make sure to bring dog treats the next time!).
    Any team member would be well-served to follow her pattern, which is to:
    \begin{enumerate}
        \item Speak softly, kindly and calmly,
            \footnote{Proverbs 25:21-22}
        \item empathize with what they seem to be feeling,
        \item apologize,
        \item and if the person continues to escalate, walk away
    \end{enumerate}
    Interactions tend to deescalate when the volunteer makes everyone feel heard, as many homeless people do not often feel that way.
    Additionally, this is an example of why it is always a good idea to walk with other members of the team.

