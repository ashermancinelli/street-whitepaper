\subsubsection{Catering to the Physical Needs}

    \qJamesPhysical
    Though the spiritual health is in every case more important than the physical, the physical is still of utmost importance, and we are commanded to tend to the physical health of those who lack it.
    The majority of Jesus' miracles were physical in nature, and Paul includes the body in the process of sanctification.
    \footnote{1 Thessalonians 5:23}
    About 15\% of Spokane County is within the poverty threshold as defined by the US Census Bureau\cite{census}, and queer people are much more likely to be within that threshold.
    According to the National Coalition for the Homeless, 43\% of clients served by drop-in centers identify as queer\cite{nch}
    There are many ways that we as the local church can tend to the needs of our community.

    \par In my experience, the most common way for the local church to provide is through bagged lunches, hygiene supplies, and clothes.
    Historically, lunches have included a sandwich, a fruit, and a snack of some sort.
    It is worth noting that many homeless people do not have healthy teeth, so they cannot eat fruits like apples or crunchy snacks.
    Additionally, foods with nut products should be kept separate in case of a nut allergy.
    Of all the clothing items to donate, socks and underwear seem to be in the highest demand.
    This is because used underwear cannot be donated and many people are not comfortable giving them out.
    The needy do greatly appreciate when we overcome this social stigma and donate unused underwear, as fewer people are willing to do so.
    The organizations and businesses along the routes also appreciate handing out goods near trash cans and/or helping pick up garbage after handing out goods.

    \qHebSix
    Each member of the Body of Christ is gifted uniquely for the construction of the Kingdom of God, and each may contribute according to their gifting.
    For example, some may be physically incapable of walking long distances, so their contribution may be the purchasing or preparing of goods to be taken with the rest of the team.
    Each ought to contribute according to their capacity, however I have met many elderly ladies, exceeding eighty years of age and with many crippling physical ailments, who faithfully take to the streets with whatever aid they can provide.
    Their steadfastness has been an encouragement to me, and I hope it will also be for any members of the Body that feel incapable of contributing.
    Let it also be an encouragement that the Father will notice and remember the efforts of every member of the body---the valiant efforts of the physically impaired are of great value to the church, and are pleasing to the Father.

