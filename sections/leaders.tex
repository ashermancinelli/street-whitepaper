
\subsection{Leaders}

    The responsibilities of team leaders exceed that of the other team members.
    Volunteers should be made aware that they may be subjected to outdated opinions, such as misogyny and racism.
    Occasionally, residents and people on the streets may speak or act out in such a way that volunteers, specifically women, are uncomfortable.
    \editonlyfn{Editors: How do I make explicit the need for male leaders and the gender specific considerations of leaders?}
    Volunteer teams tend to feel more comfortable and interactions tend to stay more appropriate when there is at least one man on any team, and leaders should ensure this is the case.

\subsubsection{Leading on Foot}

    \qLeaders
    It is usually best to have a specific route in mind before actually hitting the streets for a couple of reasons.
    \footnote{Luke 14:28-33}
    First, if a couple of members of the team are talking for a particularly long time with someone, it may be helpful for the team to temporarily split up.
    In that case, there should be at least two people equipped to lead a group, so that no group is left without anyone to lead.
    Then, because there is already an established route, each group knows where the other is going, and can regroup later.
    Additionally, a team leader should always be conscious of where his or her team members are, ideally keeping all of them within eyesight.
    Some team members may absentmindedly walk away from the group and end up alone in a crowd of people, which is undesirable.
    There is usually not any danger, however it is the team leader's responsibility to make sure that everyone stays together and is accounted for.
    \par Leaders would also do well to watch for animals, as some can be aggressive or diseased, and some team members may be allergic to or afraid of animals.
    It is usually effective to stand in between the team members and the animal, so the team members feel comfortable.

\subsubsection{Leading First-Timers}

    When leading someone downtown who has never gone before, it is extremely helpful and reassuring to describe to them what typically happens and what the plan for that day is.
    Even if they seem comfortable, they may need some encouragement to feel ready to talk to people or hand out food.
    For example, as the leader you may begin an interaction with somebody and invite the newcomer into the conversation, or ask them to give some food to someone in particular to get them started.
    Encouragement, reassurance, and gratitude go a long way in making a team member feel comfortable sharing and speaking with people on the streets.

\subsubsection{Equipping Others to Lead}

    Leading is the best way to equip another person to lead.
    Especially if someone has participated several times before, it may be helpful to let that individual carry the food/goods and handle most of the interactions with people.
    This will boost their confidence and make them feel comfortable, while you stay close by and encourage their interactions.
    The mark of a truly experienced leader is their ability to empower others to lead---in the same way, a leader downtown ought to bring up other capable leaders from within their team.

