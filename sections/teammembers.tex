
\subsection{Team Members}

    % Define what it would look like for a nonbeliever or non church member to come and be ministered to
    The following applies to volunteers from within the church, however a visiting nonbeliever or non-member may not need to bear these responsibilities.
    The volunteer may serve along with the team and be ministered to by the rest of the team.
    In this way, God's character is revealed to both the team member and the ones being served; the ministry can be both inward and outward facing.
    Otherwise, every team member should be ready for the following five responsibilities:

\subsubsection{Catering to the Physical Needs}

    \qJamesPhysical
    Though the spiritual health is in every case more important than the physical, the physical is still of utmost importance, and we are commanded to tend to the physical health of those who lack it.
    The majority of Jesus' miracles were physical in nature, and Paul includes the body in the process of sanctification.
    \footnote{1 Thessalonians 5:23}
    About 15\% of Spokane County is within the poverty threshold as defined by the US Census Bureau\cite{census}.
    There are many ways that we as the local church can tend to the needs of our community.
    In my experience, the most common way for the local church to provide is through bagged lunches, hygiene supplies, and clothes.
    Historically, lunches have included a sandwich, a fruit, and a snack of some sort.
    It is worth noting that many homeless people do not have healthy teeth, so they cannot eat fruits like apples or crunchy snacks.
    Additionally, foods with nut products should be kept separate in case of a nut allergy.
    Of all the clothing items to donate, socks and underwear seem to be in the highest demand.
    This is because used underwear cannot be donated and many people are not comfortable giving them out.
    The needy do greatly appreciate when we overcome this social stigma and donate unused underwear, as fewer people are willing to do so.
    The organizations and businesses along the routes also appreciate handing out goods near trash cans and/or helping pick up garbage after handing out goods.

    \qHebSix
    Each member of the Body of Christ is gifted uniquely for the construction of the Kingdom of God, and each may contribute according to their gifting.
    For example, some may be physically incapable of walking long distances, so their contribution may be the purchasing or preparing of goods to be taken with the rest of the team.
    Each ought to contribute according to their capacity, however I have met many elderly ladies, exceeding eighty years of age and with many crippling physical ailments, who faithfully take to the streets with whatever aid they can provide.
    Their steadfastness has been an encouragement to me, and I hope it will also be for any members of the Body that feel incapable of contributing.
    Let it also be an encouragement that the Father will notice and remember the efforts of every member of the body---the valiant efforts of the physically impaired are of great value to the church, and are pleasing to the Father.

\subsubsection{Sharing the Gospel}

    % all of this falls apart when it is not hung on the same foundation
    % the gospel is central to all of this and there is no separating this from the gospel
    \qGreatCommission
    The most capable entity in equipping believers to know and share the Gospel is the local church.
    The local church should equip its members to know and share the Gospel by biblical teaching from the pulpit, and through members teaching each other.
    Each member of a team going downtown ought be a member or attendee of a local church that professes Christ and teaches biblical doctrine, enabling the member to articulate the Gospel.
    This is a great opportunity for more mature members of the Body of Christ to model to the younger ones how to share the Gospel.
    Particularly, parents have the opportunity to show their children how to share the Gospel, a skill that may translate to their schools and eventually their workplaces.

\subsubsection{Prayer}

    In my experience, one of the methods of sharing the Gospel that has met the least friction is prayer.
    For example, ``Father in heaven, I thank you for your Gospel, which is\dots'' with a subsequent articulation of the Gospel.
    Prayer opens up a window for sharing the Gospel and is an opportunity to bring someone's needs to the Father, and every team member should be comfortable in praying with and for people they meet.

\subsubsection{Deescalation}

    Occasionally, people being served get agitated for one reason or another, and it is important to know how to deescalate these situations.
    For example, a member of a team I led once offered some cookies to a man with his dog.
    The man became quite upset that my team member only had cookies for people, and didn't bring dog treats.
    As he became more upset, my team member wisely replied, ``Well I am very sorry that we don't have any dog treats, but I will make sure to pack some next time!''
    She then proceeded to walk away from the man and wish him a nice day (she also did in fact make sure to bring dog treats the next time!).
    Any team member would be well-served to follow her pattern, which is to:
    \begin{enumerate}
        \item Speak softly, kindly and calmly,
            \footnote{Proverbs 25:21-22}
        \item empathize with what they seem to be feeling,
        \item apologize,
        \item and if the person continues to escalate, walk away
    \end{enumerate}
    Interactions tend to deescalate when the volunteer makes everyone feel heard, as many homeless people do not often feel that way.
    Additionally, this is an example of why it is always a good idea to walk with other members of the team.

\subsubsection{Developing Relationships}

    \qPiperCoronary
    It is in the context of relationship that the Holy Spirit seems to do the most work.
    \footnote{Matthew 18:20}
    In living out the Gospel, we as the Church seek not to simply distribute food to people, we seek to love them and forge relationships with them.
    This does not happen in a single day, but over weeks, months, and years.
    By going downtown many times, each team member ought to get to know and pray for specific people.
    This means remembering names and details about those we serve, with the intention of forming long-term relationships.
    Writing down names and praying over specific people, whether individually or in community, is a great way to remember details from interactions.
    This is builds relationships because remembering the names of people and details about them is indicative of intentionality and genuine care for the person.
    \editonlyfn{Editors: The previous two sentences were in the ``Prayer'' section, but was moved here. Is this the best place for it? How can this thought also be incorporated in the prayer section?}
    Referring to a person by his or her name and remembering information about them carries a special significance, especially when this demographic is often treated as sub-human.

