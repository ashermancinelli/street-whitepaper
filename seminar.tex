
%Preamble
    \documentclass[12pt]{article}
    \usepackage{setspace}
    \usepackage{arxiv}
    \usepackage{epigraph}
    \setlength{\parindent}{4em}
    \setlength{\parskip}{1em}
    \linespread{1}

    
\def \qFirstCorTwelveTwentyEight {\epigraph{And God has appointed in the church first apostles, second prophets, third teachers, then miracles, then gifts of healing, helping, administrating, and various kinds of tongues.}{\textit{1 Corinthians 12:28}}}

\def \qJamesOneTwentySeven {\epigraph{Religion that is pure and undefiled before God the Father is this: to visit orphans and widows in their affliction, and to keep oneself unstained from the world.}{\textit{James 1:27}}}

\def \qJamesTwoFive {\epigraph{5Listen, my beloved brothers, has not God chosen those who are poor in the world to be rich in faith and heirs of the kingdom, which he has promised to those who love him?}{\textit{James 2:5}}}


    \title{Street Ministry \\
        \large Equipping the Church} 

    \author{Asher Mancinelli \\
        Indian Trail Church \\
        Spokane, WA 99218 \\
        \texttt{asher.mancinelli@pnnl.gov} \\ }
%/Preamble

\begin{document}
\maketitle

\begin{abstract}
    \textit{
    By equipping the saints to lead and minister in the context of Spokane, Washington, we as the Church may better live out the Gospel.
    The local church's ministry to the homeless and food-insecure in Spokane is best accomplished by small teams with one leader, working in coordination with other long-term organizations already at work.
    The local church should not contribute to systemic problems or hinder the work of full-time ministries, but should contribute to the work of long-term organizations.
    The local church should use its financial resources to tend to the poor, and its knowledge of the Gospel to bring more people into a saving knowledge of Christ.
    }
\end{abstract}

\keywords{
    \textbf{The Church:} All those who have trusted in Christ for salvation, and who are incorporated into the Body of Christ by the Holy Spirit, spanning all time and space
    \and \textbf{The Local Church:} The local church is a group of believers in a given area who are organized under leadership, and who regularly gather together to carry out the mission of the Church
    \and \textbf{The Gospel:} That all humanity was separated from God by sin, but is overcome by an individual's faith in Christ and Christ's saving work to overcome the debt of sin, bringing that individual into communion with God and into continued sanctification and conformation into the image of Christ
}



\clearpage

\section{Problems}

    In my experience with local churches, organizations and individuals in Spokane, there are many elements that hinder those organizations and individuals from feeding and ministering to the poor, however not all of them will be addressed here.
    The three problems to be addressed are as follows:
    \begin{itemize}
        \item A lack of volunteers that feel equipped to lead
        \item A lack of understanding of what food/items should be brought
        \item Leaders are ill-equipped to coordinate teams and empower team members to talk with people
        \item Team members are ill-equipped to share the Gospel and pray with people
    \end{itemize}

%/Problems

\qJamesOneTwentySeven

\section{Intended Results}

    All team members should be equipped in the following capacities:
    \begin{itemize}
        \item Preparing bags to give away
        \item Sharing the gospel
        \item Praying with/for the served
        \item Deescalating conversations with upset people
        \item Developing relationships with those being served
    \end{itemize}

    Leaders should be equipped in the following additional capacities:
    \begin{itemize}
        \item Navigating and leading teams on foot
        \item Leading and comforting team members coming for the first time
        \item Equipping others to lead in the same capacities
    \end{itemize}

%/Intended Results

\section{Implementation}

This section describes the steps that the local church can take to achieve the goals previously outlined.

\subsection{Team Members}

    Every team member should be ready for the following five responsibilities:

\subsubsection{Preparation}

    \qJamesPhysical
    Though the spiritual health is in every case more important than the physical, the physical is still of utmost importance, and we are commanded to tend to the physical health of those who lack it.
    There are many ways that we as the local church can tend to the needs of our community.
    In my experience, the most common way for the local church to provide is through bagged lunches, hygiene supplies, and clothes.
    In the past, the lunches have had been composed of a sandwich, a fruit, and a snack.
    It is worth noting that many homeless people do not have healthy teeth, so they cannot eat fruits like apples or crunchy snacks.
    Additionally, foods with nut products should be kept separate in case of a nut allergy.
    Of all the clothing items to donate, socks and underwear seem to be in the highest demand.
    This is because used underwear cannot be donated and many people are not comfortable giving them out.
    The needy do greatly appreciate when we overcome this social stigma and donate unused underwear, as fewer people are willing to do so.

\subsubsection{Sharing the Gospel}

    \qGreatCommission
    The most capable entity in equipping believers to know and share the Gospel is the local church.
    The local church should equip its members to know and share the Gospel by biblical teaching from the pulpit, and through members teaching each other.
    Each member of a team going downtown ought be a member or attendee of a local church that professes Christ and teaches biblical doctrine, enabling the member to articulate the Gospel.
    This is a great opportunity for more mature members of the Body of Christ to model to the younger ones how to share the Gospel.
    Particularly, parents have the opportunity to show their children how to share the Gospel, a skill that may translate to their schools and eventually their workplaces.

\subsubsection{Prayer}

    In my experience, one of the methods of sharing the Gospel that has met the least friction is prayer.
    For example, ``Father in heaven, I thank you for your Gospel, which is\dots'' with a subsequent articulation of the Gospel.
    Prayer opens up a window for sharing the Gospel and is an opportunity to bring someone's needs to the Father, and every team member should be comfortable in praying with and for people they meet.

\subsubsection{Deescalation}

    Occasionally, people being served get agitated for one reason or another, and it is important to know how to deescalate these situations.
    For example, a member of a team I led offered some cookies to a man with his dog.
    The man became quite upset that my team member only had cookies for people, and didn't bring dog treats.
    As he became more upset, my team member wisely replied, ``Well I am very sorry that we don't have any dog treats, but I will make sure to pack some next time!''
    She then proceeded to walk away from the man and wish him a nice day (she also did in fact make sure to bring dog treats the next time!).
    Any team member would be well-served to follow her pattern, which is to:
    \begin{itemize}
        \item Speak softly and kindly,
        \item apologize,
        \item and if the person continues to escalate, walk away
    \end{itemize}
    Additionally, this is an example of why it is always a good idea to walk with other members of the team.

\subsubsection{Developing Relationships}

    \qPiperCoronary
    It is in the context of relationship that the Holy Spirit seems to do the most work.
    In living out the Gospel, we as the Church seek not to simply distribute food to people, we seek to love them and forge relationships with them.
    This does not happen on one day, but over weeks, months, and years.
    By going downtown many times, each team member ought to get to know and pray for specific people.
    This means remembering their name and other things they tell you, so you can form an intentional friendship with them.
    Team members would do well to write down names of people they connect with so they are able to refer to them by name next time the team member sees that person, and any other information that might help the relationship.

\subsection{Leaders}

The responsibilities of team leaders exceed that of the other team members.

\subsubsection{Navigating on Foot}

    \qTwoOrMoreAreGathered
    It is usually best to have a specific route in mind before actually hitting the streets for a couple of reasons.
    Firstly, if a couple of members of the team are talking for a particularly long time with someone, it may be helpful for the team to temporarily split up.
    In that case, there should be at least two people equipped to lead a group, so that no group is left without anyone to lead.
    Then, because there is already an established route, the each group knows where the other is going, and can regroup later.
    Additionally, a team leader should always be conscious of where his or her team members are, ideally keeping all of them within eyesight.
    Some team members may absentmindedly walk away from the group and end up alone in a crowd of people, which is undesirable.
    There is usually not any danger, however it is the team leader's responsibility to make sure that everyone stays together and is accounted for.

\subsubsection{Leading First-Timers}

    When leading someone downtown who has never gone before, it is extremely helpful and reassuring to describe to them what typically happens and what the plan for that day is.
    Even if they seem comfortable, they may need some encouragement to feel ready to talk to people or hand out food.
    You may for example start interacting with somebody and invite the newcomer into the conversation, or ask them to give some food to someone in particular to get them started.
    Encouragement, reassurance, and gratitude go a long way in making a team member feel comfortable sharing and speaking with people on the streets, I have found.

\subsubsection{Equipping Others to Lead}

    Leading is the best way to equip another person to lead.
    Especially if someone has participated several times before, it may be helpful to let that individual carry the food/goods and handle most of the interactions with people.
    This will boost their confidence and make them feel comfortable, while you stay close by and encourage their interactions.

%/Implementation

\bibliographystyle{unsrt}
\bibliography{references}

\end{document}
